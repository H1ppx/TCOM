\documentclass{article}
\usepackage[utf8]{inputenc}
\usepackage[english]{babel}
\usepackage{amsfonts}
\usepackage{amsthm}

\theoremstyle{definition}
\newtheorem{definition}{Definition}[section]
\newtheorem{theorem}{Theorem}[section]
\newtheorem{lemma}{Lemma}[section]
\renewcommand\qedsymbol{QED}

\begin{document} 

\title{Group Theory} 
\author{William Chu} 
\maketitle 

\section{Definitions}

\begin{definition}[Group]
A group is a two-tuple $(G,*)$ where $G$ is a set and $*$ is a binary operation. Also $G$ must be closed under $GxG\rightarrow G$
	\begin{enumerate}
		\item Accociativity: $\forall a,b,c \in G, a(bc) = (ab)c=a*(b*c)=(a*b)*c$
		\item Identity: $\exists e \in G$ s.t. $ea=ae=a$ $\forall a \in G$
		\item Inverses: $\forall e \in G$, $\exists a^{-1}$ s.t. $aa_{-1}= a^{-1}a =e$
	\end{enumerate}
\end{definition} 

\begin{definition}[Order of an Element]
The order of an element for any $a \in G$, $|a| = min\{n\in \mathbb{Z}^+: a^n=e\}$
\end{definition}

\begin{definition}[Order of a Group]
$|G|$ for $(G,*)$, a group, the cardinality of the set $G$.
\end{definition}

\begin{definition}[Subset]
Let $G$ be a group. The set $M$ is a subgroup of $G$ of $G$ if $M\subset G$ and $H$ is a subgroup in it's own right under the same operation denoted $H \leq G$.
\end{definition}

\section{Proofs}

\begin{theorem}
Let (G,*) be a group. Then:
	\begin{enumerate}
		\item The identity, e, is unique
		\item Inverses are unique
		\item $(a^-{1})^{-1}= a$ $\forall a \in G$
	\end{enumerate}
\end{theorem}
\begin{proof}
	\begin{enumerate}
		\item By def of $G$ being a group, the identity exists. BWOC, let $e,f \in G$, both the identity. Thus, $ef=f$ and $fe=e$ by def of identity. Hence, we also know $ef=fe$ so $f=ef=fe=e$. Thus the identity is unique.
		\item By definition of $G$, $\forall a \in G$, $\exists a^{-1}$ BWOC, let $a_1^{-1}, a_1^{-2} \in G$ are both inverses for $a$. Then, $aa_1^{-1}=e$ and $aa_2^{-1}$. So, $a_1^{-1}=a_1^{-1}e=a_1^{-1}(aa_2^{-1})=(a_1^{-1}a)a_2^{-1}ea_2^-1=a_2^{-1}$.
		\item Let a$\in G$. By definition of G, $\exists a^-1$ s.t. $aa^{-1}=e$ and $\exists(a^{-1})^{-1}$ s.t. $a^{-1}(a^{-1})^{-1}=e$.
	\end{enumerate}
\end{proof}

\end{document}