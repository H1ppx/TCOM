\documentclass{article}
\usepackage[utf8]{inputenc}
\usepackage[english]{babel}
\usepackage{amsfonts}
\usepackage{amsthm}
\usepackage{amssymb}
\usepackage{amsmath}
\usepackage[bb=boondox]{mathalfa}
\usepackage{enumitem}
 
\title{Theory of Computation (Version 2)}
\author{William Chu}
 
\theoremstyle{definition}
\newtheorem{define}{Definition}[subsection]
\newtheorem{prop}{Proposition}[subsection]
\newtheorem{lemma}{Lemma}[subsection]
\renewcommand\qedsymbol{QED}
\newcommand{\modulo}[1]{\ (\mathrm{mod}\ #1)}

 
\begin{document}
 
\maketitle
 
\tableofcontents

\section{Mathematical Preliminaries}
\subsection{Set Theory} 
\begin{define}[Set]
A set is a collection of distinct elements, where the order in which the elements are listed does not matter. The size of a set $S$, denoted $|S|$, is known as its cardinality or order. The members of a set are referred to as its elements. We denote membership of $x$ in $S$ as $x \in S$. Similarly if $x$ is not in $S$, we denote $x \notin S$.
\end{define}

\begin{define}[Set Union]
Let $A$, $B$ be sets, Then the union of $A$ and $B$, denoted $A\cup B$ is the set: $$A\cup B: = \{x: x\in A \text{ or } x \in B\}$$
\end{define}

\begin{define}[Set Intersection]
Let $A$, $B$ be sets. Then the intersection of $A$ and $B$, denoted $A\cap B$ us the set: $$A \cap B: = \{x: x\in A\text{ or }x \in B\}$$
\end{define}

\begin{define}[Symetric Difference]
Let $A$, $B$ be sets. Then the symetric difference of $A$ and $B$ , denoted $A\Delta B$is the set: $$A\Delta B := \{x:x\in A\cup B\text{, but }x\notin A\cap B\}$$
\end{define}

\begin{define}[Set Complementation]
Let $A$ be a set contained in our universe $U$. The complement of $A$, denoted $A^{C}$ or $\overline{A}$: $$ \overline{A}:= \{x\in U :x\notin A\}$$
\end{define}

\begin{define}[Set Difference]
Let $A$, $B$ be sets contained in our universe $U$. The difference of $A$ and $B$, denoted $A\backslash B$ or $A-B$ is the set: $$A\backslash B:=\{x:x\in A\text{ and }x\notin B\}$$
\end{define}

\begin{define}[Cartersian Product]
Let $A$, $B$ be sets. The cartesian product of $A$ and $B$, denoted $A\times B$, is the set: $$A\times B:=\{(a,b):a\in A\text{, }b\in B\}$$
\end{define}

\begin{define}[Subset]
Let $A$, $B$ be sets. $A$ is said to be a subset of $B$ if for every $x\in A$, we have $x\in B$ as well. This is denoted $A\subset B$ (equivocally, $A\subseteq B$). Note that $B$ is a superset of $A$.
\end{define}

\begin{define}[Power Set]
Let $S$ be a set. The power set of $S$, denoted $2^{S}$, or $\mathcal{P}(S)$, is the set of all subsets of $S$. Formally: $$2^{S}:=\{A:A\subset S\}$$
\end{define}

\begin{define}[Set equality]
Let $A$, $B$ be sets. $A$=$B$ if $A\subset B$ and $B \subset A$
\end{define}

\begin{prop}
Let $A = \{6n: n\in \mathbb{Z}\}$, $B=\{2n: n\in \mathbb{Z}\}$, $C=\{3n: n\in \mathbb{Z}\}$. So $A= B\cap C$.
\end{prop}
\begin{proof}
We first show that $A \subset B\cap C$. Let $x \in A$. By definition of $A$, $x =6k$ for some $k \in \mathbb{Z}$. We show $x \in B$ and $x \in C$. We first observe $x = 2\cdot(3k)$. Therefore, $x \in B$. Now observe$ x = 3\cdot(2k)$. So $x \in C$. Thus $x \in B\cap C$. As $x$ was arbitrairy, we conclude that $A \in B\cap C$. We now show that $(B\cap C)\subset A$. Let $y \in B\cap C$. Let $n_1, n_2 \in \mathbb{Z}$ such that $y=2n_1 = 3n_2$. As 2 and 3 share no common factors, we have that $2\cdot 3|y$. so $6|y$. Thus $y = 6h$ for some $h \in \mathbb{Z}$. So $y \in A$. Thus, $(B\cap C) \subset A$. We showed that $A \subset (B\cap C)$ and $(B\cap C)\subset A$. We conclude that $A= B\cap C$ by the definition of set equality.
\end{proof}

\begin{prop}
Let $A, B, C$ be sets, then $A\times (B\cup C) = (A\times B)\cup (A\times C)$. 
\end{prop}
\begin{proof}
We first show $A\times (B\cup C) \subset (A\times B) \cup (A\times C)$. Let $(x,y) \in A \times (B\cup C)$. We show that $(x,y)\in (A\times B)\cup (A\times C)$. Suppose $y\in B$. Then $(x,y)\in A\times B$. Otherwise, $y\in C$. Then $(x,y)\in A\times C$. So $(x,y) \in (A\times B)$ or $(x,y)\in A\times C$. Thus, $(x,y)\in (A\times B) \cup (A\times C)$. We conclude that $A\times (B\cup C)\subset (A\times B)\cup (A\times C)$. We now show $(A\times B)\cup (A\times C) \subset A \times(B\cup C)$. Let $(x,y)\in (A \times B)\cup (A\times C)$. We show that $(x,y) \in A\times(B\cup C)$. Suppose $(x,y)\in A \times B$. So $x \in A$ and $y \in B$. Thus, $(x,y)\in A \times (B\cup C)$. Otherwise, $(x,y)\in A\times C$. So $x \in A$ and $y \in C$. So $y \in B\cup C/ Thus, (x,y)\in A\times(B\cup C)$. Since in both cases, $(x,y)\in A\times(B \cup C)$, we conclude that $(A\times B)\cup(A\times C)\subset A\times (B\cup C)$. Since $A \subset (B\cap C)$ and $(B\cap C)\subset A$ and $(A\times B)\cup(A\times C)\subset A\times (B\cup C)$, $A\times (B\cup C) = (A\times B)\cup (A\times C)$.
\end{proof}

\subsection{Relations}
TODO: FINISH

\begin{define}[Relation] 
Let $X$ be a set. A $k$-ary relation on $X$ is a subset $R \subset X^k$.
\end{define}

\begin{define}[Function]
Let $X$ and $Y$ be sets. A function $f$ is a subset (or 1-place relation) of $X \times Y$ such that for every $x \in X$, $\exists! y \in Y$ where $(x,y) \in f$.
\end{define}

\begin{define}[Injection] 
A function $f : X \to Y$ is said to be an injection if $f(x_1) = f(x_2) \Rightarrow x_1 = x_2$. Equivocally, $f$ is an injection if $x_1 \neq x_2 \Rightarrow f(x_1) \neq f(x_2)$.
\end{define}

\begin{define}[Surjection]
 Let $X$ and $Y$ be sets. A function $f : X \rightarrow Y$ is a surjection if $\forall y \in Y$, $\exists x \in X$ s.t. $f(x) = y$.
\end{define}

\begin{define}[Bijection]
Let $X$ and $Y$ be sets. A bijection is a function$ f : X \rightarrow Y$ that is both an injection and a surjection.
\end{define}

\begin{define}[Reflexive Relation]
A relation $R$ on the set $X$ is said to be reflexive if $(a, a) \in R$ for every $a \in X$.
\end{define}

\begin{define}[Symmetric Relation]
A relation $R$ on the set $X$ is said to be symmetric if $(a, b) \in R$ if and only if $(b,a) \in R$ for every $a,b \in X$.
\end{define}

\begin{define}[Transitive Relation]
A relation $R$ on the set $X$ is said to be transitive if for every $a, b, c \in X$ satisfying $(a, b), (b, c) \in R,$ then $(a, c) \in R$.
\end{define}

\begin{define}[Equivalence Relation]
An equivalence relation is a reflexive, symmetric, and transitive relation.
\end{define}

\begin{define}[Congruence Relation]
Let $n \leq 1$ be an integer. The congruence relation modulo $n$ is a binary relation on $\mathbb{Z}$ given by: $a \equiv b (\text{mod }n)$ (read as: a is congruent to b modulo n) if and only if $n|(b-a)$.
\end{define}

\begin{prop}
The congruence relation modulo $n$ is an equivelence relation.
\end{prop}
\begin{proof}
We show that the congruence relation modulo n is reflexive, symmetric, and transitive.
\begin{itemize}[align=left]
\item[Reflexivity:] We show that $\forall a \in \mathbb{Z}$, $a \equiv a \modulo{n}$. Observe that $a-a=0$. So $n\times 0 = 0 = a-a$. Thus $n|a-a$. So $a \equiv a \modulo{n}$. We conclude that the congruence relation modulo $n$ is reflexive.

\item[Symmetry:] We show that $b \equiv a \modulo{n}$. Let $q \in \mathbb{Z}$ such that $nq = a-b$. Thus, $n(-q) = b-a$, so $n|(b-a)$. Thus, $b \equiv a \modulo{n}$. So the congruence relation modulo $n$ is symmetric.

\item[Transitive:] Let $a,b,c \in \mathbb{Z}$ s.t. $a \equiv b \modulo{n}$ and $b \equiv c \modulo{n}$. We show that $a \equiv c \modulo{n}$. By the definition of the congruence relation, $n|(a-b)$ and $n|(b-c)$. Let $h, k \in \mathbb{Z}$ s.t. $nh=a-b$ and $nk=b-c$. So $nh+nk=n(h+k)=a-c$. Thus, $n|(a-c)$, so $a\equiv c \modulo{n}$. It follows that the congruence relation modulo $n$ is transitive.
\end{itemize}
We conclude that the congruence relation modulo n is an equivalence relation.
\end{proof}



\subsection{Induction}
TODO

\subsection{Asymptotics}

\subsection{Combinatorics}
TODO

\subsection{Countability}
TODO

\subsection{Graph Theory}
TODO


\section{Autonoma Theory}

\subsection{Regular Expressions}
TODO
 
 \subsection{Finite State Automata}
TODO

\subsection{Pumpng Lemma}
TODO

\subsection{Closure Propertoes}
TODO

\subsection{Myhill-Nerode and DFA Minimization}
TODO

\section{Group Theory}
TODO

\subsection{Brzozowski Algebraic Method}
TODO

\subsection{Dihedral Groups}
TODO

\subsection{Symmetry Groups}
TODO

\section{Turing Machines and Decidability}

\subsection{Standard Deterministic Turing Machine}
TODO

\subsection{Undecidability}
TODO

\subsection{Reducibility}
TODO

\section{Complexity Theory}

\subsection{$\mathcal{P}$ and $\mathcal{NP}$}
TODO

\subsection{$\mathcal{NP}$-Completeness}
TODO

\end{document}