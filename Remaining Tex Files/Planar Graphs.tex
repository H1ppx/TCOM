\documentclass{article}
\usepackage[utf8]{inputenc}
\usepackage[english]{babel}
\usepackage{amsfonts}
\usepackage{amsthm}

\theoremstyle{definition}
\newtheorem{definition}{Definition}[section]
\newtheorem{theorem}{Theorem}[section]
\newtheorem{lemma}{Lemma}[section]
\renewcommand\qedsymbol{QED}

\begin{document} 

\title{Planar Graphs} 
\author{William Chu} 
\maketitle 

\section{Definitions}

\begin{definition}[Deterministic Turing Machine]
A graph $G(V, E)$ is  said to be planar, if there exists an injection $\varphi: V(G)\rightarrow\mathbb{R}$, s.t. if $uv, xy \in E(G)$, then the simple curves $[\varphi(u), \varphi(v)]$ and $[\varphi(x), \varphi(y)]$ do not intersect, except possibly at the endpoints (and only if two endpoints are the same)
\end{definition}

\section{Proofs}

\begin{theorem}
Let G(V,E) be a simple, connected, planar graph, with $n = |V|$, $m = |E|$ and $F$ faces. Then: $n-m+f = 2$.
\end{theorem}
\begin{proof}
By induction on $m \in \mathbb{N}$, the number of edges. Base Case: Suppose $m=0$. As $G$ is connected $G \cong K_1$. So $n = 1$ and $f = 1$. So $n-m+f=1-0+1=2$. Inductive Hypothesis: Fix $m \leq 0$, and suppose that every plane connected graph on at most $m$ edges satisfies $n-m+f =2$. Inductive Step: Consider a connected, planar graph on $m+1$ edges.
\begin{itemize}
\item Case 1: Suppose $G$ has a vertex of degree 1, $V$. Observe that $G`$ has $n-1$ vertices and $m-1$ edges, By the inductive hypothesis, $G`$ satisfies $(n-1)-(m-1)+f=2$. Thus, G satisfies: $n-m+f=2$.
\item Case 2: Suppose all vertices have degree at least 2. Thus, G has a cycle, We remove a single edge e on the boundary of some cycle. Note that e touches 2 faces. Removing e decreases the number of faces by 1. So by the inductive hypothesis, $G/\{e\}$ satisfies: $n-(m-1)+(f-1)=2$. Thus $G$ satisfies: $n-m+f=2$.
\end{itemize}
\end{proof}


\end{document}