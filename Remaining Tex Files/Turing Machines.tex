\documentclass{article}
\usepackage[utf8]{inputenc}
\usepackage[english]{babel}
\usepackage{amsfonts}
\usepackage{amsthm}

\theoremstyle{definition}
\newtheorem{definition}{Definition}[section]
\newtheorem{theorem}{Theorem}[section]
\newtheorem{lemma}{Lemma}[section]
\renewcommand\qedsymbol{QED}

\begin{document} 

\title{Turing Machine} 
\author{William Chu} 
\maketitle 

\section{Definitions}

\begin{definition}[Deterministic Turing Machine]
A deterministic turing machine is a 7-tuple: (Q, $\Sigma$, $\Gamma$, $\delta$, $q_0$, $q_{accept}$, $q_{reject}$) where:
	\begin{enumerate}
		\item Q is the finite set of states
		\item $\Sigma$ is the input alphabet
		\item $\Gamma$ is the tape alphabet, where $\Sigma \subset \Gamma$ and the black symbol $\beta \in \Gamma$
	\end{enumerate}
\end{definition}

\begin{definition}[Recusively Enumerable]
A language $L$ is recusively enumerable if there exists a turing machine $M$ s.t. $L(M)=L$
\end{defintion}

\begin{definition} ;
A language $L$ is decidable if $L$ is recusively enumerable and $M$ halts on all such strings.

\end{document}